\documentclass[11pt,a4paper]{article}

% Basic packages
\usepackage[margin=1in]{geometry}
\usepackage{amsmath,amssymb}
\usepackage{graphicx}
\usepackage{hyperref}
\usepackage{parskip}

% Title information
\title{Notes}
\author{Martin}
\date{\today}

% Custom commands for quick note-taking
\newcommand{\todo}[1]{\textbf{TODO: #1}}
\newcommand{\note}[1]{\textit{Note: #1}}
\newcommand{\important}[1]{\underline{#1}}

\begin{document}

\maketitle

\section{Structures and thermal}
% Quick notes go here
Ch14 P401
\begin{enumerate}
    \item The structure will mainly support payload during launch and a stable platform for measurements
    \item Optical payloads require clean environments
\end{enumerate}
Ch14 P417
\begin{enumerate}
    \item Spacecraft natural frequencies must be separate from launch vehicle ones.
    \item Load isolation systems have become more common in recent years. Flexures/Dampers
    \item Solar panel shock loads are a concern
    \item For precision pointing instruments, graphite composites do not change size over temperature. "Effective optical bench structure uses graphite-composite face-sehets over an aluminium honeycomb core". light and does not change over tempreature.
\end{enumerate}
Ch22 P663
P668 basically tells me cad
\begin{enumerate}
    \item Make sure each componenet can be assembled nicely to each other
    \item Make sure everything is within fairing
    \item Model FOV of cameras and sensors, and CG
    \item CG limits
\end{enumerate}
CH22 P668
15\% to 20\% of the mass of the spacecraft is the structure. 
Load analysis can only really be done later on, to look at sinusoidal loading, quasistatic loading, and random loading.

\section{Micrometeorites}
Bumper sizing
\begin{equation}
    t_b = c_b d \frac{\rho_p}{\rho_b}
\end{equation}
Where
\begin{itemize}
    \item $c_b = $ coefficient 0.25 when S/d < 30, $c_b$ = 0.2 when above
    \item $d$ projeoctile diameter (cm)
    \item $\rho_p$ = projectile density (g/cm$^3$)
    \item $\rho_b$ = bumper density (g/cm$^3$)
    \item $S$ = spacing between outer and rear wall (cm)
    \item $t_b$ = bumper thickness (cm)
\end{itemize}


Rear wall sizing

\begin{equation}
    t_w = c_w d^{0.5} (\rho_b \rho_p)^{\frac{1}{6}}(M_p)^{\frac{1}{3}}\frac{V_n}{S^{0.5}}(\frac{70}{\sigma})^{0.5}
\end{equation}
Where
\begin{itemize}
    \item $c_w$ = 0.16 cm$^2$-sec (g$^2/3$ km)
    \item $d$ = projectile diameter (cm)
    \item $M_p$ = projectile mass (g)
    \item $\rho_p$ = projectile density (g/cm$^3$)
    \item $\rho_b$ = bumper density (g/cm$^3$)
    \item $S$ = spacing between outer and rear wall (cm)
    \item $\sigma$ = yield strength of rear wall material (ksi)
    \item $t_w$ = rear wall thickness (cm)
    \item $V_n$ = normal velocity of projectile (km/sec)
\end{itemize}

\end{document}